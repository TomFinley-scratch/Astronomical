% Documentation for NGC 2000.0 (Ed. by R.W. Sinnott 1988)
%
% Document Number: NSSDC/WDC-A-R&S 89-29
%
% Written: December 1989 by Wayne H. Warren
% TeX Ver: April 1991 by Lee E. Brotzman
%
\documentstyle [11pt]{article}
%
\newfont{\bfit}{cmbxti10}
\newcommand{\dmv}{Documentation for the Machine-Readable Version of}
\newcommand{\cat}{NGC 2000.0}
\newcommand{\abbr}{7118}
\newcommand{\prefp}{Sinnott 1988}
\newcommand{\adc}{Astronomical Data Center}
\newcommand{\docnum}{NSSDC/WDC-A-R\&S 89-29}
%
\begin{document}
%
\setcounter{page}{0}
\thispagestyle{empty}
\null \vfil \vskip 1in
\begin{center}
\begin{tabular}{c}
      {\LARGE \cat} \\
      \\
      {\Large (Edited by R.W. Sinnott 1988)} \\
      \\
      Documentation for the Computer-Readable Version \\
      \\
      Wayne H. Warren Jr. \\
      December 1989 \\
      \\
      {\small Doc. No. \docnum} \\
      \\
      \\
      {\small National Space Science Data Center (NSSDC)/} \\
      {\small World Data Center A for Rockets and Satellites (WDC-A-R\&S)} \\
      {\small National Aeronautics and Space Administration} \\
      {\small Goddard Space Flight Center} \\
      {\small Greenbelt, MD 20771}
\end{tabular}
\end{center}
\vfil \null
\pagebreak

\begin{abstract}

The machine-readable version of the catalog, as it is currently being
distributed from the \adc, is described. {\em \cat} is a modern version of the
NGC and IC catalogs compiled by J. L. E. Dreyer in the late nineteenth and early
twentieth centuries. Errata compiled by Dreyer and by subsequent workers have
been incorporated into the new version and the object types have been updated
with information from modern astronomy; the descriptions given are those of
Dreyer. The order of the new catalog is strictly by right ascension, the NGC and
IC objects being merged into one machine-readable file.

\end{abstract}

\section{Introduction}

\subsection{Description}

{\em \cat} is a modern compilation of the {\em New General Catalogue of Nebulae
and Clusters of Stars} (NGC), the {\em Index Catalogue} (IC), and the {\em
Second Index Catalogue} compiled by J. L. E. Dreyer (1888, 1895, 1908). The new
compilation of these classical catalogs is intended to meet the needs of
present-day observers by reporting positions at equinox 2000.0 and by
incorporating the corrections reported by Dreyer himself and by a host of other
astronomers who have worked with the data and compiled lists of errata. The
object types given are those known to modern astronomy.

This document describes the machine-readable version of {\em \cat} as it is
currently being distributed from the \adc\ (ADC). It includes descriptions of
the data and format of the computerized version so that users of the file can
process the data without problems and guesswork. It is, however, not intended to
replace the source reference, which gives much more complete information
regarding the compilation of the modern version, sources of the corrections
incorporated into the file, information about the object descriptions used and
their abbreviations, nomenclature, and statistics of object locations according
to constellation.

{\bf All users of the machine version are urged to acquire a copy of the
published book in order to have complete information at their disposal. However,
a copy of this document should be transmitted to any recipient of the
machine-readable catalog.}

{\bfit This catalog is copyrighted by Sky Publishing Corporation, which has
kindly deposited the machine version in the data centers for permanent archiving
and dissemination to astronomers for scientific research purposes only. The data
should not be used for commercial purposes without the explicit permission of
Sky Publishing Corporation.}

\subsection{Source Reference}

\begin{itemize}
\item {\em \cat, The Complete New General Catalogue and Index Catalogue of
Nebulae and Star Clusters by J. L. E. Dreyer}, ed. R. W. Sinnott 1988 (Sky
Publishing Corporation and Cambridge University Press).
\end{itemize}

\section{Structure}

\subsection{File Summary}

The machine version of {\em \cat} consists of two files. The first file contains
only a brief statement concerning the fact that the catalog is copyrighted. The
second file contains all of the data in the catalog. Table \ref{t:sumtab} gives
the machine-independent file attributes. All logical records are of fixed
length, and if the catalog is received on magnetic tape, it will contain blocks
of fixed length (as noted below) except that the last block of the data file may
be short. (The text file consists of a single 320-byte block.)

\begin{table}[htbp]
\centering
\begin{tabular}{|c|l|c|c|r|}
\hline
\multicolumn{5}{|c|}{\cat\/ (\prefp)}                             \\
\hline
     &                               & Record & Record &
                                     \multicolumn{1}{c|}{Number of} \\
File & \multicolumn{1}{c|}{Contents} & Format & Length &
                                     \multicolumn{1}{c|}{Records}   \\
\hline
  1  & Text & Fixed & 80  & 4     \\
  2  & Data & Fixed & 100 & 13226 \\
\hline
\end{tabular}
\caption{Summary Description of Catalog Files}
\label{t:sumtab}
\end{table}

The information contained in the above table is sufficient for a user to
describe the indigenous characteristics of the machine-readable version of
{\em \cat} to a computer. Information easily varied from installation to
installation, such as block size (physical record length), blocking factor
(number of logical records per physical record), total number of blocks,
density, number of tracks and character coding (ASCII, EBCDIC) for tapes, is not
included, but should always accompany secondary copies if any are supplied to
other users or installations.

\subsection{Text File (File 1 of 2)}
\label{s:cathdi}

This file contains only a brief statement concerning copyright in a Fortran A80
format. A single line of text with a copyright statement was originally included
as the first record in the data file, but this was moved to a separate file and
supplemented with additional text at the ADC.

\subsection{Catalog (File 2 of 2)}
\label{s:cathdc}

This file contains the complete catalog almost exactly as it appears on pages
1-241 of the source reference. The file is composed of a single logical record
per object and the textual fields of the records contain upper and lower case
characters as in the book.

Table \ref{t:cattab} gives a byte-by-byte description of the contents of the
data file. A suggested Fortran format specification for reading each data field
is included and can be modified depending upon individual programming and
processing requirements (Fortran 77 character string-type formats are used);
however, caution is advised when substituting format specifications because some
data fields contain character data and others are blank when data are absent.
For such numerical fields, primary real format specifications are given to
indicate decimal-point locations, while alternate A-type formats are specified
in parentheses. The numerical fields that can be empty are the size and
magnitude data and these cannot have valid zero values in this catalog, so the
data can be read with the primary format and tested for zero to detect missing
data. Default (null) values are always blanks in data fields for which primary
suggested formats are given as A. Null values are not specified for numerical
fields that always contain valid data.

\begin{table}[htbp]
\centering
\begin{tabular}{|c|c|c|c|l|}
\hline
       &       & Suggested & Default &                            \\
 Bytes & Units &   Format  &  Value  & \multicolumn{1}{c|}{Data}  \\
\hline
  1-8  & ---    &   A8     &   ---   & Object identification     \\
  9    & ---    &   1X     &   ---   & Blank                     \\
 10-12 & ---    &   A3     &   ---   & Object type               \\
 13    & ---    &   1X     &   ---   & Blank                     \\
 14-15 & hours  &   I2     &   ---   & Right ascension           \\
 16    & ---    &   1X     &   ---   & Blank                     \\
 17-20 & min    &   F4.1   &   ---   & R.A.                      \\
 21-22 & ---    &   2X     &   ---   & Blank                     \\
 23    & ---    &   A1     &   ---   & Sign of declination       \\
 24-25 & deg    &   I2     &   ---   & Declination               \\
 26    & ---    &   1X     &   ---   & Blank                     \\
 27-28 & arcmin &   I2     &   ---   & Dec.                      \\
 29    & ---    &   1X     &   ---   & Blank                     \\
 30    & ---    &   A1     &   ---   & Modern data source code   \\
 31-32 & ---    &   2X     &   ---   & Blank                     \\
 33-35 & ---    &   A3     &   ---   & Constellation             \\
 36    & ---    &   A1     &   ---   & Upper limit character ($<$) \\
 37-41 & arcmin & F5.1 (A5)&  blank  & Object size               \\
 42-43 & ---    &   2X     &   ---   & Blank                     \\
 44-47 & mag    & F4.1 (A4)&  blank  & Magnitude                 \\
 48    & ---    &   A1     &   ---   & Magnitude code            \\
 49    & ---    &   1X     &   ---   & Blank                     \\
 50-99 & ---    &   A50    &   ---   & Description               \\
 100   & ---    &   1X     &   ---   & Blank                     \\
\hline
\end{tabular}
\caption{Data File Record Format}
\label{t:cattab}
\end{table}

\begin{description}%{23ex}

\item[Object ID] Object number in the form ``NGC nnnn'' for NGC objects, and
``IC nnnn'' for IC objects, where ``nnnn'' indicates the sequential number of
the object.

\item[Object type] An object classification according to modern astronomy. The
field is coded according to the following abbreviations:

\begin{description}%{7ex}
\item[Gx] Galaxy
\item[OC] Open star cluster
\item[Gb] Globular star cluster, usually in the Milky Way Galaxy
\item[Nb] Bright emission or reflection nebula
\item[Pl] Planetary nebula
\item[C+N] Cluster associated with nebulosity
\item[Ast] Asterism or group of a few stars
\item[Kt] Knot or nebulous region in an external galaxy
\item[***] Triple star
\item[D*] Double star
\item[*] Single star
\item[?] Uncertain type or may not exist
\item[blank] Unidentified at the place given, or type unknown
\item[--] Object called nonexistent in the RNGC (Sulentic and Tifft 1973)
\item[PD] Photographic plate defect
\end{description}

\item[Equatorial coordinates] Equinox 2000.

\item[Data source code] A letter code to indicate the source of modern data
about the object. These citations will be found in the source reference, pages
XXIII-XXIV, along with additional information. ``Modern'' data may include type,
position, size, and magnitude, but not descriptions, which are always those of
Dreyer.

\item[Constellation] Constellation in which the object is located.

\item[Upper limit character] The character ``$<$'' is present if object size is
an upper limit.

\item[Object size] Angular size, as measured along the greatest dimension. The
precision varies, so byte 41 can be blank, as can the whole field if size is not
reported.

\item[Magnitude] Integrated (total) magnitude of the type indicated by the code
in the following field. The precision varies as in the size field.

\item[Magnitude code] Blank if the integrated magnitude is visual, ``p'' if it
is a photographic (blue) magnitude.

\item[Description] A description of the object, as given by Dreyer or corrected
by him, in a coded or abbreviated form. For an NGC object, the description is
always a visual impression, while the IC descriptions are often based on
photographic appearance. A full list of the abbreviations will be found in Table
II of the introduction to the published catalog (the source reference).

\end{description}

\section{History}

\subsection{Remarks and Modifications}

It is important, even for users of the machine-readable catalog and this
documentation, to also have a copy of the published book. In addition to the
tables and reference sources mentioned in this document, the book provides an
introductory section with a brief history of the NGC and IC catalogs, a count of
objects by constellation, information on Dreyer's descriptions, a table cross
index of Messier and NGC/IC designations, and a table of common names for NGC
objects. The book also contains a table of right ascensions for NGC and IC
objects.

A magnetic tape containing \cat\ was received from William E. Shawcross of Sky
Publishing Corporation on August 14, 1989. According to Mr. Shawcross, the file
supplied to the ADC was an unmodified version of the one used to produce the
book, and it still contained the \TeX\ commands employed to produce the special
symbols present in the printed version. As received, the file also contained a
single copyright text record at its beginning. The text record was removed to an
added first file in the archived version and supplemented with a small amount of
additional information. The \TeX\ in the data file was replaced by standard
characters to represent the information. Special symbols, such as $\Delta$,
$\bigcirc$, etc., were changed to their spelled-out equivalents.

The size field was modified to add decimal points to integer numbers and to
align all values properly so that the field can be processed with a single
format specification. The magnitude field was modified by moving the ``p'' code
for photographic magnitude to its own byte in order to remove it from the
numerical field. Decimal points were added to all integer numbers in this field
also.

The catalog data file was run through the ADC General Verification Program,
which checks data ranges and for various other problems that can be detected in
a systematic way.

\section{Acknowledgments and References}

\subsection{Acknowledgments}

Appreciation is expressed to William E. Shawcross for responding to a request
from the ADC to make {\em \cat} available to the scientific community in
machine-readable form. Mr. Shawcross also arranged for a copy of the
machine-readable \TeX\ file to be created for deposit in the archives of the
data centers. I am grateful to both Mr. Shawcross and to Roger W. Sinnott for
reviewing a draft copy of this document and making comments. The comments
resulted in the finding and elimination of a few \TeX\ symbols that were missed
during the initial work.

\subsection{References}

\begin{itemize}

\item Dreyer, J. L. E. 1888, {\em New General Catalogue of Nebulae and Clusters
of Stars}, {\em Mem. Roy. Astron. Soc.} {\bf 49}, Part I (reprinted 1953,
London: Royal Astronomical Society).

\item Dreyer, J. L. E. 1895, {\em Index Catalogue of Nebulae Found in the Years
1888 to 1894, with Notes and Corrections to the New General Catalogue}, {\em
Mem. Roy. Astron. Soc.} {\bf 51}, 185 (reprinted 1953, London: Royal
Astronomical Society).

\item Dreyer, J. L. E. 1908, {\em Second Index Catalogue of Nebulae Found in the
Years 1895 to 1907; with Notes and Corrections to the New General Catalogue and
to the Index Catalogue for 1888 to 1894}, {\em Mem. Roy. Astron. Soc.} {\bf 59},
Part 2, 105 (reprinted 1953, London: Royal Astronomical Society).

\item {\em \cat, The Complete New General Catalogue and Index Catalogue of
Nebulae and Star Clusters by J. L. E. Dreyer}, ed. R. W. Sinnott 1988 (Sky
Publishing Corporation and Cambridge University Press).

\item Sulentic, J. W. and Tifft, W. G. 1973, {\em The Revised New General
Catalogue of Nonstellar Astronomical Objects} (Tucson: The University of Arizona
Press).

\end{itemize}

\end{document}
